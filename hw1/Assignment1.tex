% Options for packages loaded elsewhere
\PassOptionsToPackage{unicode}{hyperref}
\PassOptionsToPackage{hyphens}{url}
%
\documentclass[
]{article}
\usepackage{amsmath,amssymb}
\usepackage{lmodern}
\usepackage{iftex}
\ifPDFTeX
  \usepackage[T1]{fontenc}
  \usepackage[utf8]{inputenc}
  \usepackage{textcomp} % provide euro and other symbols
\else % if luatex or xetex
  \usepackage{unicode-math}
  \defaultfontfeatures{Scale=MatchLowercase}
  \defaultfontfeatures[\rmfamily]{Ligatures=TeX,Scale=1}
\fi
% Use upquote if available, for straight quotes in verbatim environments
\IfFileExists{upquote.sty}{\usepackage{upquote}}{}
\IfFileExists{microtype.sty}{% use microtype if available
  \usepackage[]{microtype}
  \UseMicrotypeSet[protrusion]{basicmath} % disable protrusion for tt fonts
}{}
\makeatletter
\@ifundefined{KOMAClassName}{% if non-KOMA class
  \IfFileExists{parskip.sty}{%
    \usepackage{parskip}
  }{% else
    \setlength{\parindent}{0pt}
    \setlength{\parskip}{6pt plus 2pt minus 1pt}}
}{% if KOMA class
  \KOMAoptions{parskip=half}}
\makeatother
\usepackage{xcolor}
\usepackage[margin=1in]{geometry}
\usepackage{color}
\usepackage{fancyvrb}
\newcommand{\VerbBar}{|}
\newcommand{\VERB}{\Verb[commandchars=\\\{\}]}
\DefineVerbatimEnvironment{Highlighting}{Verbatim}{commandchars=\\\{\}}
% Add ',fontsize=\small' for more characters per line
\usepackage{framed}
\definecolor{shadecolor}{RGB}{248,248,248}
\newenvironment{Shaded}{\begin{snugshade}}{\end{snugshade}}
\newcommand{\AlertTok}[1]{\textcolor[rgb]{0.94,0.16,0.16}{#1}}
\newcommand{\AnnotationTok}[1]{\textcolor[rgb]{0.56,0.35,0.01}{\textbf{\textit{#1}}}}
\newcommand{\AttributeTok}[1]{\textcolor[rgb]{0.77,0.63,0.00}{#1}}
\newcommand{\BaseNTok}[1]{\textcolor[rgb]{0.00,0.00,0.81}{#1}}
\newcommand{\BuiltInTok}[1]{#1}
\newcommand{\CharTok}[1]{\textcolor[rgb]{0.31,0.60,0.02}{#1}}
\newcommand{\CommentTok}[1]{\textcolor[rgb]{0.56,0.35,0.01}{\textit{#1}}}
\newcommand{\CommentVarTok}[1]{\textcolor[rgb]{0.56,0.35,0.01}{\textbf{\textit{#1}}}}
\newcommand{\ConstantTok}[1]{\textcolor[rgb]{0.00,0.00,0.00}{#1}}
\newcommand{\ControlFlowTok}[1]{\textcolor[rgb]{0.13,0.29,0.53}{\textbf{#1}}}
\newcommand{\DataTypeTok}[1]{\textcolor[rgb]{0.13,0.29,0.53}{#1}}
\newcommand{\DecValTok}[1]{\textcolor[rgb]{0.00,0.00,0.81}{#1}}
\newcommand{\DocumentationTok}[1]{\textcolor[rgb]{0.56,0.35,0.01}{\textbf{\textit{#1}}}}
\newcommand{\ErrorTok}[1]{\textcolor[rgb]{0.64,0.00,0.00}{\textbf{#1}}}
\newcommand{\ExtensionTok}[1]{#1}
\newcommand{\FloatTok}[1]{\textcolor[rgb]{0.00,0.00,0.81}{#1}}
\newcommand{\FunctionTok}[1]{\textcolor[rgb]{0.00,0.00,0.00}{#1}}
\newcommand{\ImportTok}[1]{#1}
\newcommand{\InformationTok}[1]{\textcolor[rgb]{0.56,0.35,0.01}{\textbf{\textit{#1}}}}
\newcommand{\KeywordTok}[1]{\textcolor[rgb]{0.13,0.29,0.53}{\textbf{#1}}}
\newcommand{\NormalTok}[1]{#1}
\newcommand{\OperatorTok}[1]{\textcolor[rgb]{0.81,0.36,0.00}{\textbf{#1}}}
\newcommand{\OtherTok}[1]{\textcolor[rgb]{0.56,0.35,0.01}{#1}}
\newcommand{\PreprocessorTok}[1]{\textcolor[rgb]{0.56,0.35,0.01}{\textit{#1}}}
\newcommand{\RegionMarkerTok}[1]{#1}
\newcommand{\SpecialCharTok}[1]{\textcolor[rgb]{0.00,0.00,0.00}{#1}}
\newcommand{\SpecialStringTok}[1]{\textcolor[rgb]{0.31,0.60,0.02}{#1}}
\newcommand{\StringTok}[1]{\textcolor[rgb]{0.31,0.60,0.02}{#1}}
\newcommand{\VariableTok}[1]{\textcolor[rgb]{0.00,0.00,0.00}{#1}}
\newcommand{\VerbatimStringTok}[1]{\textcolor[rgb]{0.31,0.60,0.02}{#1}}
\newcommand{\WarningTok}[1]{\textcolor[rgb]{0.56,0.35,0.01}{\textbf{\textit{#1}}}}
\usepackage{graphicx}
\makeatletter
\def\maxwidth{\ifdim\Gin@nat@width>\linewidth\linewidth\else\Gin@nat@width\fi}
\def\maxheight{\ifdim\Gin@nat@height>\textheight\textheight\else\Gin@nat@height\fi}
\makeatother
% Scale images if necessary, so that they will not overflow the page
% margins by default, and it is still possible to overwrite the defaults
% using explicit options in \includegraphics[width, height, ...]{}
\setkeys{Gin}{width=\maxwidth,height=\maxheight,keepaspectratio}
% Set default figure placement to htbp
\makeatletter
\def\fps@figure{htbp}
\makeatother
\setlength{\emergencystretch}{3em} % prevent overfull lines
\providecommand{\tightlist}{%
  \setlength{\itemsep}{0pt}\setlength{\parskip}{0pt}}
\setcounter{secnumdepth}{-\maxdimen} % remove section numbering
\ifLuaTeX
  \usepackage{selnolig}  % disable illegal ligatures
\fi
\IfFileExists{bookmark.sty}{\usepackage{bookmark}}{\usepackage{hyperref}}
\IfFileExists{xurl.sty}{\usepackage{xurl}}{} % add URL line breaks if available
\urlstyle{same} % disable monospaced font for URLs
\hypersetup{
  pdftitle={R Notebook},
  hidelinks,
  pdfcreator={LaTeX via pandoc}}

\title{R Notebook}
\author{}
\date{\vspace{-2.5em}}

\begin{document}
\maketitle

\begin{Shaded}
\begin{Highlighting}[]
\CommentTok{\#To set path to the working directory, please paste your path below where 04cars.rda dataset is located and uncomment the code below}
\CommentTok{\#setwd("D:/projects/R\_Projects/Sandbox/Stats\_model\_assignments") }
\CommentTok{\#getwd()}
\end{Highlighting}
\end{Shaded}

\hypertarget{question-1}{%
\subsubsection{Question 1}\label{question-1}}

Write a function confBand(x, y, conf=0.95) taking in a predictor vector
(x1, . . . , xn) and a response vector y = (y1,\ldots,yn) and return a
plot with the points (x1,y1),\ldots,(xn,yn), the least squares line, and
the confidence band at level conf. Apply your function to hp and mpg
from the 04cars dataset.

\begin{Shaded}
\begin{Highlighting}[]
\CommentTok{\#Loading the cars dataset}
\FunctionTok{load}\NormalTok{(}\StringTok{"04cars.rda"}\NormalTok{)}
\NormalTok{data }\OtherTok{=}\NormalTok{ dat[,}\FunctionTok{c}\NormalTok{(}\DecValTok{13}\NormalTok{,}\DecValTok{15}\NormalTok{)]}
\FunctionTok{names}\NormalTok{(data) }\OtherTok{=} \FunctionTok{c}\NormalTok{(}\StringTok{"hp"}\NormalTok{, }\StringTok{"mpg"}\NormalTok{)}

\CommentTok{\#defining the confBand function}
\NormalTok{confBand }\OtherTok{\textless{}{-}} \ControlFlowTok{function}\NormalTok{(x,y, }\AttributeTok{conf =} \FloatTok{0.95}\NormalTok{)\{}
  
\NormalTok{  n }\OtherTok{=} \DecValTok{100}
  \CommentTok{\#Plotting the linear model}
\NormalTok{  linear\_model }\OtherTok{=} \FunctionTok{lm}\NormalTok{(y}\SpecialCharTok{\textasciitilde{}}\NormalTok{x)}
\NormalTok{  p}\OtherTok{=}\DecValTok{1}
\NormalTok{  df\_1 }\OtherTok{=}\NormalTok{ p}\SpecialCharTok{+}\DecValTok{1} \CommentTok{\#degree of freedom 1}
\NormalTok{  df\_2}\OtherTok{=}\NormalTok{n}\SpecialCharTok{{-}}\NormalTok{p}\DecValTok{{-}1}  \CommentTok{\#degree of freedom 2}
\NormalTok{  fquartile }\OtherTok{=} \FunctionTok{sqrt}\NormalTok{((p}\SpecialCharTok{+}\DecValTok{1}\NormalTok{) }\SpecialCharTok{*} \FunctionTok{qf}\NormalTok{(conf, df\_1,df\_2))}
  
  \CommentTok{\#Calculating y bar}
\NormalTok{  y\_ }\OtherTok{=} \FunctionTok{predict}\NormalTok{(linear\_model, }\FunctionTok{data.frame}\NormalTok{(}\AttributeTok{x=}\NormalTok{x), }\AttributeTok{se =}\NormalTok{  T)}
  
  \CommentTok{\#Calculating the upper and lower confidence bounds}
\NormalTok{  cb\_upper }\OtherTok{=}\NormalTok{  (y\_}\SpecialCharTok{$}\NormalTok{fit }\SpecialCharTok{+}\NormalTok{ ((fquartile)}\SpecialCharTok{*}\NormalTok{y\_}\SpecialCharTok{$}\NormalTok{se.fit))}
\NormalTok{  cb\_lower }\OtherTok{=}\NormalTok{  (y\_}\SpecialCharTok{$}\NormalTok{fit }\SpecialCharTok{{-}}\NormalTok{ ((fquartile)}\SpecialCharTok{*}\NormalTok{y\_}\SpecialCharTok{$}\NormalTok{se.fit))}
  

  \FunctionTok{plot}\NormalTok{(x,y, }\AttributeTok{type =} \StringTok{\textquotesingle{}p\textquotesingle{}}\NormalTok{)}
  
  \FunctionTok{lines}\NormalTok{(x,cb\_upper, }\AttributeTok{type =} \StringTok{\textquotesingle{}l\textquotesingle{}}\NormalTok{, }\AttributeTok{col =}\StringTok{"green"}\NormalTok{)}
  \FunctionTok{lines}\NormalTok{(x,cb\_lower, }\AttributeTok{type =} \StringTok{\textquotesingle{}l\textquotesingle{}}\NormalTok{, }\AttributeTok{col =}\StringTok{"green"}\NormalTok{)}
  \FunctionTok{abline}\NormalTok{(linear\_model, }\AttributeTok{col =} \StringTok{\textquotesingle{}red\textquotesingle{}}\NormalTok{, }\AttributeTok{lwd =} \DecValTok{2}\NormalTok{)}
  \FunctionTok{legend}\NormalTok{(}\StringTok{"topright"}\NormalTok{, }\AttributeTok{legend=}\FunctionTok{c}\NormalTok{(}\StringTok{"Linear model"}\NormalTok{, }\StringTok{"Conf band"}\NormalTok{),}
       \AttributeTok{col=}\FunctionTok{c}\NormalTok{(}\StringTok{"red"}\NormalTok{, }\StringTok{"green"}\NormalTok{), }\AttributeTok{lty=}\DecValTok{1}\SpecialCharTok{:}\DecValTok{2}\NormalTok{, }\AttributeTok{cex=}\FloatTok{0.8}\NormalTok{, }\AttributeTok{text.font=}\DecValTok{4}\NormalTok{, }\AttributeTok{bg=}\StringTok{\textquotesingle{}001000\textquotesingle{}}\NormalTok{)}
  
  \CommentTok{\#return (list(upb = cb\_upper, lwb = cb\_lower, y\_pred = y\_$fit, pred\_se = y\_$se.fit, Fvalue = fquartile))}
\NormalTok{\}}

\FunctionTok{confBand}\NormalTok{(data}\SpecialCharTok{$}\NormalTok{hp, data}\SpecialCharTok{$}\NormalTok{mpg)}
\end{Highlighting}
\end{Shaded}

\includegraphics{Assignment1_files/figure-latex/unnamed-chunk-2-1.pdf}

\hypertarget{question-2}{%
\subsubsection{Question 2}\label{question-2}}

Let n = 100 and draw x1, . . . , xn ∼ Unif(0, 1), which stay fixed in
what follows. Repeat the following experiment N = 1000 times.

• Generate yi = 1 + xi + εi, with εi i.i.d. N (0, 0.2). • Compute the
99\% confidence band and record whether it contains the true line, or
not.

Summarize the result of this numerical experiment by returning the
proportion of times (out of N) that the confidence band contained the
true line.

\begin{Shaded}
\begin{Highlighting}[]
\NormalTok{x }\OtherTok{=} \FunctionTok{runif}\NormalTok{(}\DecValTok{100}\NormalTok{, }\DecValTok{0}\NormalTok{,}\DecValTok{1}\NormalTok{)}
\NormalTok{y\_true }\OtherTok{=} \FloatTok{1.000}\SpecialCharTok{+}\NormalTok{x}
\NormalTok{contains\_true\_line }\OtherTok{=} \FunctionTok{rep}\NormalTok{(}\ConstantTok{NA}\NormalTok{, }\DecValTok{1000}\NormalTok{)}

\CommentTok{\# looping 1000 times}
\ControlFlowTok{for}\NormalTok{ (j }\ControlFlowTok{in} \DecValTok{1}\SpecialCharTok{:}\DecValTok{1000}\NormalTok{)\{}
  
  \CommentTok{\#calculating the error with N(0,0.2)}
\NormalTok{  e }\OtherTok{=} \FunctionTok{rnorm}\NormalTok{(}\DecValTok{100}\NormalTok{, }\DecValTok{0}\NormalTok{, }\FloatTok{0.2}\NormalTok{)}
\NormalTok{  y }\OtherTok{=} \FloatTok{1.000}\SpecialCharTok{+}\NormalTok{x}\SpecialCharTok{+}\NormalTok{e }\CommentTok{\#generating y for every j}
  
\NormalTok{  CI }\OtherTok{=} \FunctionTok{data.frame}\NormalTok{(}\AttributeTok{lower =} \FunctionTok{rep}\NormalTok{(}\ConstantTok{NA}\NormalTok{,}\DecValTok{100}\NormalTok{),}\AttributeTok{upper =} \FunctionTok{rep}\NormalTok{(}\ConstantTok{NA}\NormalTok{,}\DecValTok{100}\NormalTok{),}\AttributeTok{contain\_true\_value =} \FunctionTok{rep}\NormalTok{(}\ConstantTok{NA}\NormalTok{,}\DecValTok{100}\NormalTok{))}
  \CommentTok{\#fitting the linear model}
\NormalTok{  linear\_model}\OtherTok{=} \FunctionTok{lm}\NormalTok{(y}\SpecialCharTok{\textasciitilde{}}\NormalTok{x)}
  
\NormalTok{  n }\OtherTok{=} \DecValTok{100}
\NormalTok{  p}\OtherTok{=}\DecValTok{1}
\NormalTok{  df\_1 }\OtherTok{=}\NormalTok{ p}\SpecialCharTok{+}\DecValTok{1}
\NormalTok{  df\_2}\OtherTok{=}\NormalTok{n}\SpecialCharTok{{-}}\NormalTok{p}\DecValTok{{-}1}
\NormalTok{  conf }\OtherTok{=} \FloatTok{0.99}
\NormalTok{  F\_quartile }\OtherTok{=} \FunctionTok{sqrt}\NormalTok{((p}\SpecialCharTok{+}\DecValTok{1}\NormalTok{) }\SpecialCharTok{*} \FunctionTok{qf}\NormalTok{(conf, df\_1,df\_2))}
\NormalTok{  pred\_op }\OtherTok{=} \FunctionTok{predict}\NormalTok{(linear\_model, }\FunctionTok{data.frame}\NormalTok{(}\AttributeTok{hp=}\NormalTok{x), }\AttributeTok{se =}\NormalTok{  T)}
  

\NormalTok{  CI}\SpecialCharTok{$}\NormalTok{lower }\OtherTok{=}\NormalTok{ (pred\_op}\SpecialCharTok{$}\NormalTok{fit }\SpecialCharTok{{-}}\NormalTok{ ((F\_quartile)}\SpecialCharTok{*}\NormalTok{pred\_op}\SpecialCharTok{$}\NormalTok{se.fit))}
\NormalTok{  CI}\SpecialCharTok{$}\NormalTok{upper }\OtherTok{=}\NormalTok{ (pred\_op}\SpecialCharTok{$}\NormalTok{fit }\SpecialCharTok{+}\NormalTok{ ((F\_quartile)}\SpecialCharTok{*}\NormalTok{pred\_op}\SpecialCharTok{$}\NormalTok{se.fit))}
\NormalTok{  CI}\SpecialCharTok{$}\NormalTok{contain\_true\_value }\OtherTok{=}\NormalTok{ (CI}\SpecialCharTok{$}\NormalTok{lower }\SpecialCharTok{\textless{}=}\NormalTok{y\_true }\SpecialCharTok{\&}\NormalTok{ CI}\SpecialCharTok{$}\NormalTok{upper }\SpecialCharTok{\textgreater{}=}\NormalTok{ y\_true) }
  
  \CommentTok{\#contains\_true\_line[j] = ((mean(CI$contain\_true\_value)) ==1)}
\NormalTok{  contains\_true\_line[j] }\OtherTok{=}\NormalTok{ ((}\FunctionTok{sum}\NormalTok{(CI}\SpecialCharTok{$}\NormalTok{contain\_true\_value)) }\SpecialCharTok{==}\NormalTok{n)}
\NormalTok{\}}


\CommentTok{\#answer = (sum(contains\_true\_line)/1000)}
\NormalTok{answer }\OtherTok{=}\NormalTok{ (}\FunctionTok{sum}\NormalTok{(contains\_true\_line)}\SpecialCharTok{/}\DecValTok{1000}\NormalTok{)}

\FunctionTok{sprintf}\NormalTok{(}\StringTok{\textquotesingle{}Result of the experiment \%f\textquotesingle{}}\NormalTok{, answer)}
\end{Highlighting}
\end{Shaded}

\begin{verbatim}
## [1] "Result of the experiment 0.993000"
\end{verbatim}

\end{document}
